\documentclass{report}
\usepackage{ucs}
\usepackage[utf8x]{inputenc}
\usepackage{hyperref}
\usepackage[english,romanian]{babel}

\title{{\sc Raport asupra practicii: 25.06-06.07.2018}}
\author{\underline{Andreea Stan}}
\date{\textbf{\textit{Specializare: Informatică, anul 2, grupa 4\\  Lector dr. Andrei Rusu}}}
\begin{document}
\maketitle

\tableofcontents

\chapter{Introducere}

In aceste zile de practică ,voi implementa \textit{algoritmul} , \textbf{heap sort}, \textit{de sortare prin stivă}. 
\vskip 0.5cm
Pentru implementarea algoritmului voi folosi sistemul de operare \textbf{\textit{Android}}.
\vskip 0.5cm
\textbf{Codul aplicatiei se gasește pe contul de GitHub :} \\
\url{https://github.com/stanAndreea/Practica-Facultate}

\chapter{Activități planificate}
\begin{enumerate}

\item  Luni, 25.06.2018 \newline
Aducerea la cunoștință a obiectivelor și cerințelor practicii de producție.
\item  Marți, 26.06.2018 \newline
Configurarea sistemelor software pe calculatoare. 
\item  Miercuri, 27.06.2018 \newline
Studierea modului de lucru cu GitHub, Android.
\item  Joi, 28.06.2018 \newline
Studierea și practicarea LaTeX, BitBucket. Interfețe grafice de lucru cu Git (SmartGit).
\item  Vineri, 29.06.2018  \newline
Inițierea unei lucrări (descrierea unui algoritm, a unei teme agreate cu prof. coordonator)
\item  Luni, 02.07.2018  \newline
Lucrul asupra lucrării - Interfața grafică
\item  Marți, 03.07.2018  \newline
Lucrul asupra lucrării - Interfața grafică + Background
\item  Miercuri, 04.07.2018  \newline
Lucrul asupra lucrării - Background
\item  Joi, 05.07.2018  \newline
Prezentarea lucrărilor
\item  Vineri, 06.07.2018  \newline
Notarea finală a activității
\end{enumerate}
\chapter{25.06.2018}
Am discutat cu prof. coordonator despre cum vor decurge orele de practică , ce vom studia.

\chapter{26.06.2018}
Am desfăţurat următoarele activităţi:
\begin{itemize}
\item
Am identificat sursele pentru MikTeX, GitHub, Android pentru realizarea proiectului.
\begin{itemize}
\item
Am instalat, configurat pe calculatorul de lucru aplicațiile necesare:
\begin{itemize}
\item
MikTeX
\item
Android Studio 
\end{itemize}
\end{itemize}
\end{itemize}

\chapter{27.06.2018}

\vskip 0.5cm
Studierea obiectivelor și cerințelor față de practica de producție. Clarificarea situațiilor incerte.
\begin{itemize}
    \item  Am studiat modul de lucru cu GitHub , Latex.
    \item Am citit documentația de la Android.
\end{itemize}

\vskip 0.5cm
\textbf{ Android} este o platformă software și un sistem de operare pentru dispozitive și telefoane mobile bazată pe nucleul \textit{Linux}, dezvoltată inițial de compania \textit{Google}, iar mai târziu de consorțiul comercial \textit{ Open Handset Alliance}.\textbf{Android} permite dezvoltatorilor să scrie cod gestionat în limbajul Java, controlând dispozitivul prin intermediul bibliotecilor \textbf{Java} dezvoltate de \textit{Google}.
\vskip 0.5cm
\textbf{GitHub} este un serviciu de găzduire web pentru proiecte de dezvoltare a software-ului care utilizează sistemul de control al versiunilor Git. GitHub oferă planuri tarifare pentru depozite private, și conturi gratuite pentru proiecte open source. Site-ul a fost lansat în 2008.
\vskip 0.5cm
\textbf{LaTeX} ste un limbaj de programare de nivel-înalt, util în a accede la toate resursele limbajului \textbf{TeX}. Deoarece TeX este un limbaj de programare de nivel scăzut s-a dovedit a fi destul de dificil de utilizat de către utilizatorii comuni, motiv pentru care LaTeX a fost construit special pentru a permite oricărui utilizator să beneficieze de puterea limbajului TeX.

\chapter{28.06.2018}
\begin{itemize}
    \item Am inceput sa lucrez în Latex .
    \item \textbf{Suplimentar} am studiat si un pic despre \textbf{BitBucket} si \textbf{SmarGit}
\end{itemize}

\chapter{29.06.2018}
\begin{itemize}
    \item Am ales tema pentru proiect : \textbf{\textit{Implementarea algoritmului de sortare prin stivă (heap sort)}} 
    \item Am studiat modul de acționare a sortări de tip \textbf{Heap Sort}
    \item Am început să lucrez la interfața grafică pentru proiect.
\end{itemize}
\chapter{02.07.2018}
\begin{itemize}
    \item Am terminat interfața grafica a proiectului .
    \item Am început să lucrez la partea de background a aplicației.
\end{itemize}
\chapter{03.07.2018}
\begin{itemize}
    \item Am citit despre \textbf{Heap Sort} din materialele oferite de prof. coordonator.
    \item Am implementat algoritmul \textbf{Heap Sort}.
\end{itemize}
\chapter{04.07.2018}
\begin{itemize}
    \item Am finalizat proiectul .
    \item Am terminat de scris raportul.
    \item Am facut o ultima verificare a proiectului.
\end{itemize}
\chapter{05.07.2018}
\begin{itemize}
    \item Prezentarea proiectului.
\end{itemize}
\chapter{06.07.2018}
Notarea finală a activității.

\chapter{Concluzii}
Am invățat să lucrez cu Latex, GitHub, BitBucket, Android și am învățat despre \textbf{Heap Sort}.
\begin{itemize}
    \item Un heap binar este un arbore binar cu următoarele proprietăţi:
    \begin{itemize}
        \item este „complet“(toate nivelele sunt pline, cu posibila excepţie a ultimului nivel), adică de înălţime minimă
        \item există aceeaşi relaţie de ordine între orice nod şi părintele acestuia(excepţie - nodul rădăcină).
    \end{itemize}
    \item Dacă nodurile conţin numere întregi după care stabilim relaţia de ordine, heap-ul poate fi de două feluri:
    \begin{itemize}
        \item max-heap(rădăcina are cel mai mare număr, de la orice copil la părinte avem relaţia mai mic sau egal)
        \item min-heap(rădăcina are cel mai mic număr, de la orice copil la părinte avem relaţia mai mare sau egal).
    \end{itemize}
\end{itemize}
\end{document}